%------------------ vorlage.tex ------------------------------------------------
%
% LaTeX-Vorlage zur Erstellung von Projektdokumentationen
% im Fachbereich Informatik der Hochschule Trier
%
% Basis: Vorlage 'svmono' des Springer Verlags
% Bearbeiter: Hermann Schloß, Christian Bettinger
%
%-------------------------------------------------------------------------------


%------------------ Präambel ---------------------------------------------------
\documentclass[envcountsame, envcountchap, deutsch]{i-studis}

\usepackage[utf8]{inputenc}

\usepackage[a4paper]{geometry}
\usepackage[english, ngerman]{babel}

\usepackage[pdftex]{graphicx}
\usepackage{epstopdf}

\usepackage{listings}

\usepackage[german, ruled, vlined]{algorithm2e}
\usepackage{amssymb, amsfonts, amstext, amsmath}
\usepackage{array}
\usepackage[skip=10pt]{caption}
\usepackage[usenames, dvipsnames]{color}
\usepackage[pdftex, plainpages=false]{hyperref}
\usepackage{textcomp}
\usepackage{tabularx}

\usepackage{bibgerm}
\bibliographystyle{geralpha}

\usepackage{makeidx}
\usepackage{multicol}
\makeindex

\usepackage{tikz}
\usetikzlibrary{shapes.geometric, arrows.meta, positioning, shadows}
\usepackage{eurosym}
\usepackage{amsmath}
\usepackage[T1]{fontenc}
\usepackage{graphicx}

\pagestyle{myheadings}
\setlength{\textheight}{1.1\textheight}

\lstset{
	basicstyle=\scriptsize\ttfamily,
	commentstyle=\scriptsize\ttfamily\color{Gray},
	identifierstyle=\scriptsize\ttfamily,
	keywordstyle=\scriptsize\ttfamily,
	stringstyle=\scriptsize\ttfamily,
	tabsize=4,
	numbers=left,
	numberstyle=\tiny,
	numberblanklines=false,
	frame=single,
	framesep=3mm,
	framexleftmargin=7mm,
	xleftmargin=10mm,
	linewidth=144mm,
	captionpos=b,
}


%------------------ Manuelle Silbentrennung ------------------------------------
\hyphenation{Ele-men-tar-ob-jek-te ab-ge-tas-tet Aus-wer-tung House-holder-Matrix Least-Squares-Al-go-ri-th-men}


%------------------ Titelseite -------------------------------------------------
\begin{document}

\title{Erklärung von Klassifikatoren des maschinellen Lernens durch vielfältige kontrafaktische Erklärungen}
\subtitle{Explaining Machine Learning Classifiers through Diverse Counterfactual Explanations}

\author{Fabian Wagner}

\supervisor{Prof. Dr. Jörn Schneider, Marvin Schneider}

\address{Trier}
\submitdate{11.02.2026}

%------------------ Projektart -------------------------------------------------
%\project{Bachelor-Teamprojekt}
%\project{Bachelor-Abschlussarbeit}
%\project{Master-Projektstudium}
%\project{Master-Abschlussarbeit}
\project{Seminar}
%\project{Hausarbeit}

\mytitlepage

%------------------ Vorwort, Kurzfassung, Verzeichnisse ------------------------
\frontmatter
%\input{chapters/Vorwort}								% Vorwort (optional)
\kurzfassung

%In der Kurzfassung soll in kurzer und prägnanter Weise der wesentliche Inhalt der Arbeit beschrieben werden. Dazu zählen vor allem eine kurze Aufgabenbeschreibung, der Lösungsansatz sowie die wesentlichen Ergebnisse der Arbeit. Ein häufiger Fehler für die Kurzfassung ist, dass lediglich die Aufgabenbeschreibung (d.h. das Problem) in Kurzform vorgelegt wird. Die Kurzfassung soll aber die gesamte Arbeit widerspiegeln. Deshalb sind vor allem die erzielten Ergebnisse darzustellen. Die Kurzfassung soll etwa eine halbe bis ganze DIN-A4-Seite umfassen.
%
%Hinweis: Schreiben Sie die Kurzfassung am Ende der Arbeit, denn eventuell ist Ihnen beim Schreiben erst vollends klar geworden, was das Wesentliche der Arbeit ist bzw. welche Schwerpunkte Sie bei der Arbeit gesetzt haben. Andernfalls laufen Sie Gefahr, dass die Kurzfassung nicht zum Rest der Arbeit passt.
%
%\kurzfassungEN
%
%The same in English.

KI-Systeme treffen immer mehr Entscheidungen die das Leben von Menschen beeinflussen, wie Gewährung von Krediten oder die Klassifikation von Bewerbungen. Umso wichtiger ist die Nachvollziehbarkeit und eine Erklärung wieso diese Entscheidung getroffen wurde und was geändert werden muss, um ein anderes Ergebnis zu erhalten. Ein möglicher Erklärungsansatz, welcher konkrete Handlungsanweisungen aufzeigen kann, ist Diverse Counterfactual Explanations (DiCE). Hierbei werden diverse Gegenbeispiele generiert, um dem Anwender verschiedene Möglichkeiten zum Lösen des Problems zu geben.

Die Methodik von DiCE verwendet eine kombinierte Verlustfunktion zur gleichzeitigen Optimierung von Gültigkeit, Proximität und Diversität. Während Proximität die Ähnlichkeit zur ursprünglichen Eingabe sicherstellt, bietet Diversität dem Anwender möglichst unterschiedliche Handlungsanweisungen. Es ist zu beachten, dass die beiden Konzepte hierbei konkurrierende Ziele verfolgen. Die Sparsität minimiert die Anzahl der geänderten Features in einem Nachbearbeitungsschritt, sodass die Machbarkeit der Beispiele zu erhöhen.

Die Evaluation des Verfahrens wird anhand von vier Datensätzen und dem Vergleich mit vier Baselines und LIME. DiCE erreicht eine nahezu 100\%-ige Gültigkeit. Die Baselines fallen bei steigender Zahl an zu generierenden Beispielen stark ab. Die lokale Entscheidungsgrenze kann durch DiCE ebenfalls besser als LIME annähern. In der Praxis ist DiCE jedoch stark von der Vorarbeit der Anwender durch explizite Angabe von Randbedingungen abhängig, um hilfreiche Erklärungen zu generieren.							% Kurzfassung/Abstract
\tableofcontents										% Inhaltsverzeichnis
%\listoffigures											% Abbildungsverzeichnis (optional)
%\listoftables											% Tabellenverzeichnis (optional)
%\lstlistoflistings										% Listings (optional)


%------------------ Kapitel ----------------------------------------------------
\mainmatter
\chapter{Einleitung und Problemstellung}

%todo prüfe welche englischen begriffe übersetzt werden sollen!!!!
%todo ausfürhlicher mit beispielen
%todo einarbeiten des Papers 0 in die Einleitung und zitieren aus diesem!
Künstliche Intelligenz etabliert sich zu einem festen Bestandteil der Arbeitswelt und dem Alltag. Neben dem Einsatz als Chatbot oder Assisstenten treffen KI-Systeme vermehrt Entscheidungen, welche folgenschwere Konsequenzen auf das Leben haben. Die Nachvollziehbarkeit einer solchen Entscheidung ist für den sicheren und fairen Einsatz von KIs von zentraler Bedeutung. Erklärbare KI (xAI) befasst sich mit der Fragestellung die getroffene Klassifikation einer KI für den Menschen verständlich und nachvollziehbar zu machen.

Ziel der Seminararbeit ist die Analyse und Evaluation der Methodik von DiCE \cite{mothilal2020dice} sowie die Bewertung für die Anwendbarkeit der Methode in der Praxis. Der Ansatz von DiCE beruht auf der Erklärung durch Beispiele, die ähnlich zur ursprünglichen Eingabe sind, jedoch nicht zur gleichen Klassifikation durch die KI führen. Als erstes werden grundlegende Begriffe erklärt, welche für das Verständnis der Arbeit notwendig sind. Anschließend wird das methodische Vorgehen von DiCE untersucht. Hierzu werden die Konzepte Diversity, Proximity, Sparsity, User Constraints sowie Feasibility erläutert und deren Abhängigkeiten betrachtet. Die Bewertung von DiCE erfolgt mithilfe von Evaluationsmetriken für die einzelnen Konzepte, welche zuvor definiert werden. Weiterhin wird ein Beispiel demonstriert, an dem die Verständlichkeit einer solchen Erklärung durch Counterfactuals verdeutlicht werden soll. Abschließend folgt eine Zusammenfassung und kritische Diskussion der vorgestellten Methodik.


%
%Begonnen werden soll mit einer Einleitung zum Thema, also Hintergrund und Ziel erläutert werden.
%
%Weiterhin wird das vorliegende Problem diskutiert: Was ist zu lösen, warum ist es wichtig, dass man dieses Problem löst und welche Lösungsansätze gibt es bereits. Der Bezug auf vorhandene oder eben bisher fehlende Lösungen begründet auch die Intention und Bedeutung dieser Arbeit. Dies können allgemeine Gesichtspunkte sein: Man liefert einen Beitrag für ein generelles Problem oder man hat eine spezielle Systemumgebung oder ein spezielles Produkt (z.B. in einem Unternehmen), woraus sich dieses noch zu lösende Problem ergibt.
%
%Im weiteren Verlauf wird die Problemstellung konkret dargestellt: Was ist spezifisch zu lösen? Welche Randbedingungen sind gegeben und was ist die Zielsetzung? Letztere soll das beschreiben, was man mit dieser Arbeit (mindestens) erreichen möchte.

\chapter{Theoretische Grundlagen}

In diesem Kapitel werden zunächst wichtige Begriffe aus dem Bereich der erklärbaren KI vorgestellt und in dem Kontext von DiCE erläutert.


\section{Grundlegende Begriffe}

Das Framework DiCE wird primär im Bereich der überwachten Klassifikation eingesetzt, weshalb der Fokus der folgenden Definitionen auf diesem Teilbereich des maschinellen Lernens liegt.

\subsection{Formaler Rahmen der Klassifikation}
Im Kontext dieser Arbeit wird ein \textbf{Modell des maschinellen Lernens} als eine Funktion betrachtet, die Eingabedaten einer bestimmten Kategorie zuordnet. Die Eingabe erfolgt in Form eines \textbf{Feature-Vektors}, welcher die numerischen und kategorischen Merkmale eines Objekts zusammenfasst. Das Ergebnis der Verarbeitung ist die Zuordnung zu einer diskreten \textbf{Klasse}, wie in Abbildung \ref{fig:ml-prozess} exemplarisch gezeigt wird. 
Während ML-Modelle auch für kontinuierliche Vorhersagen genutzt werden können, konzentriert sich die Generierung von Counterfactuals auf die Veränderung der Klassenzugehörigkeit.
\begin{figure}[htbp]
	\centering
	\resizebox{\textwidth}{!}{
	\begin{tikzpicture}[
		node distance=1.5cm,
		auto,
		databox/.style={
			rectangle, 
			draw=blue!50!black!80, 
			fill=blue!5!white,
			thick, 
			rounded corners, 
			text width=3.8cm, % Etwas breiter für den Vektor
			align=center, 
			minimum height=4em, % Höher für vertikalen Vektor
			font=\small
		},
		blackbox/.style={
			rectangle, 
			draw=black!80, 
			fill=gray!20,
			thick, 
			text width=3.5cm, 
			align=center, 
			minimum height=4em,
			font=\bfseries\small,
			drop shadow
		},
		myarrow/.style={
			->, 
			>={Latex[width=3mm,length=3mm]}, 
			thick,
			draw=gray!80
		}
		]
		
		% --- Knoten ---
		
		% Der Input mit vertikalem Spaltenvektor
		\node[databox] (input) {
			\textbf{Feature-Vektor}\\
			\vspace{0.2cm}
			$\begin{bmatrix}
				200\,\text{\euro} \\
				\text{Männlich} \\
				\text{Arbeitslos}
			\end{bmatrix}$
		};
		
		\node[blackbox, right=of input] (model) {
			ML-Modell \\ 
			(Black-Box)
		};
		
		\node[databox, right=of model] (output) {
			\textbf{Klasse} \\
			\vspace{0.2cm}
			\glqq Kredit abglehnt\grqq
		};
		
		% --- Pfeile ---
		\draw[myarrow] (input) -- (model);
		\draw[myarrow] (model) -- (output);
		
	\end{tikzpicture}
	}
	\caption{Visualisierung der Datenverarbeitung: Eine ursprüngliche Eingabe als Feature-Vektor wird durch das ML-Modell einer Klasse zugeordnet. Das ML-Modell lehnt den Antrag auf einen Kredit von einer männlichen, arbeitslosen Person mit einem Einkommen von 200\euro{} ab.}
	\label{fig:ml-prozess}
\end{figure}


\subsection{Kontrafaktische Erklärungen}
Eine \textbf{kontrafaktische Erklärung}, auch Counterfactual (CF) genannt, setzt dort an, wo die ursprüngliche Eingabe zu einer aus Anwendersicht unerwünschten Klasse führt. Ein Counterfactual ist ein modifiziertes Gegenbeispiel zur ursprünglichen Eingabe, welches so gewählt wird, dass das Modell eine \textbf{erwünschte Klasse} (CF-Klasse) ausgibt. 
Das Ziel ist es, dem Anwender die minimal notwendigen Änderungen aufzuzeigen, um ein gewünschtes Ergebnis zu erzielen.\cite{mothilal2020dice}
\begin{figure}[htbp]
	\centering
	\resizebox{\textwidth}{!}{
	\begin{tikzpicture}[
		node distance=1.5cm,
		auto,
		databox/.style={
			rectangle, 
			draw=blue!50!black!80, 
			fill=blue!5!white,
			thick, 
			rounded corners, 
			text width=3.8cm, % Etwas breiter für den Vektor
			align=center, 
			minimum height=4em, % Höher für vertikalen Vektor
			font=\small
		},
		blackbox/.style={
			rectangle, 
			draw=black!80, 
			fill=gray!20,
			thick, 
			text width=3.5cm, 
			align=center, 
			minimum height=4em,
			font=\bfseries\small,
			drop shadow
		},
		myarrow/.style={
			->, 
			>={Latex[width=3mm,length=3mm]}, 
			thick,
			draw=gray!80
		}
		]
		
		% --- Knoten ---
		
		% Der Input mit vertikalem Spaltenvektor
		\node[databox] (input) {
			\textbf{Counterfactual} \\
			\vspace{0.2cm}
			$\begin{bmatrix}
				4000\,\text{\euro} \\
				\text{Männlich} \\
				\text{Beamter}
			\end{bmatrix}$
		};
		
		\node[blackbox, right=of input] (model) {
			ML-Modell \\ 
			(Black-Box)
		};
		
		\node[databox, right=of model] (output) {
			\textbf{CF-Klasse} \\
			\vspace{0.2cm}
			\glqq Kredit gewährt\grqq
		};
		
		% --- Pfeile ---
		\draw[myarrow] (input) -- (model);
		\draw[myarrow] (model) -- (output);
		
	\end{tikzpicture}
	}
	\caption{Beispielhafte kontrafaktische Erklärung im Bezug auf Abbildung \ref{fig:ml-prozess}, sodass durch Variation der Eingabe die gewünschte Klasse erreicht wird. Das Counterfacutal zeigt, dass eine Erhöhung des Einkommens und eine Anstellung den gewünschten Kredit durch das Modell ermöglichen würde.}
	\label{fig:ml-prozess-cf}
\end{figure}

\subsection{Einordnung von Erklärungsmodellen}
Die Notwendigkeit für Verfahren wie DiCE ergibt sich aus der Komplexität moderner \textbf{Black-Box}-Modelle. Im Gegensatz zu \textbf{White-Box}-Modellen ist deren interne Logik nicht unmittelbar nachvollziehbar\cite{8882211}. 
Erklärungsansätze lassen sich in zwei Kategorien, global und lokal, unterteilen. \textbf{Globale} Ansätze versuchen, das Modellverhalten in seiner Gesamtheit zu interpretieren, während \textbf{lokale} Ansätze, zu denen auch die kontrafaktischen Erklärungen zählen, die Entscheidung für einen spezifischen Einzelfall begründen\cite{barredo2019explainable}.


\section{Definition von DiCE}
% DiCE: post-hoc, black-box, lokaler Ansatz
\textbf{Di}verse \textbf{C}ounterfactual \textbf{E}xplanations (DiCE) ist ein Erklärungsansatz, um Entscheidungen von ML-Modellen für den Menschen verständlich zu machen. Es handelt sich um ein lokales, post-hoc Verfahren für Black-Box-Modelle. Die Erklärung des Modells findet somit nach der Trainingsphase statt und dient lediglich der Bewertung, Nachvollziehbarkeit sowie dem Aufzeigen von Schwächen oder einem Bias in den Ausgaben des Modells.
DiCE generiert zu einer Eingabe verschiedene Counterfactuals, um dem Anwender aufzuzeigen, welche Parameter sich für eine andere Klassifikation durch das ML-Modell ändern müssen. Die Counterfactuals werden so generiert, dass sie möglichst unterschiedlich sind, dabei jedoch möglichst nahe an der ursprünglichen Eingabe bleiben.\cite{mothilal2020dice}



%==============================
%% ML-Modell
%Ein \textbf{Modell des maschinellen Lernens} (ML-Modell) ist ein algorithmisches System, das mit Trainingsdaten zur Erkennung von Mustern durch überwachtes, unüberwachtes, halbüberwachtes oder bestärkendes Lernen trainiert wurde. Anschließend kann das Modell unbekannte Eingabedaten auf Basis des gelernten Wissens verarbeiten und klassifizieren.
%
%% Input: Feature Vector
%Die Eingabe erfolgt in Form eines sogenannten \textbf{Feature-Vektors}, wobei die Dimension gleich der Anzahl an kategorischen und numerischen Merkmalen ist.
%% Output: Klasse
%Das Ergebnis erfolgt über die Angabe einer \textbf{Klasse}, zu der eine Eingabe durch das ML-Modell zugeordnet wurde.
%% Original Input / Output? Eigentlich selbsterklärend
%Die \textbf{ursprüngliche Eingabe} bezeichnet den unveränderten Feature Vector, welcher in einer \textbf{unerwünschten Klasse} resultiert. 
%% CF / CF Klasse
%Eine \textbf{kontrafaktische Erklärung}, auch \textbf{Counterfactual} (CF) genannt, ist ein verändertes Gegenbeispiel der Eingabe, welches so gewählt wird, dass nicht die ursprünglich erhaltene, unerwünschte Klasse das Ergebnis ist, sondern eine andere Klasse. Diese erwünschte Klasse wird als \textbf{CF Klasse} bezeichnet.\cite{mothilal2020dice}
%
%% Black-Box vs. White-Box-Modelle
%Bei ML-Modellen wird zwischen \textbf{Black-Box-} und \textbf{White-Box-Modellen} unterschieden. 
%Black-Box-Modelle sind schwer erklärbar und für Domänenexperten unverständlich, sodass ein Ergebnis nicht nachvollziehbar für einen Menschen ist. Als White-Box-Modelle werden hingegen ML-Modelle bezeichnet, die erklärbare und nachvollziehbare Resultate liefern.\cite{8882211}
%
%% Lokale vs. Globale Erklärungsansätze
%Ein \textbf{globaler} Erklärungsansatz verfolgt das Ziel, das gesamte Entscheidungsverhalten eines ML-Modells zu verstehen. Im Gegensatz dazu verfolgen \textbf{lokale} Ansätze das Ziel ein Modell in einem eingeschränkten, weniger komplexen Lösungsraum zu erklären.\cite{barredo2019explainable}

%=================================


% Problem
% Ist das nicht in der Einleitung schon beschrieben oder muss das nochmal hier hin?

% Ansatz zum Lösen des Problems
					% Grundlegende Begriffe und Terminologien
\chapter{Methodisches Vorgehen}
In diesem Kapitel werden die zentralen Konzepte von DiCE betrachtet, welche zur Generierung der Counterfactuals benötigt werden. Darunter fallen Proximität, Diversität und Sparsität. Abschließend wird die Verlustfunktion als Möglichkeit der Optimierung von Counterfactuals untersucht.


\section{Zentrale Konzepte}
%- Diversity via dpp
\textbf{Diversität} beschreibt, wie sich generierten Counterfactuals voneinander unterscheiden.
Eine hohe Vielfältigkeit zeigt dem Anwender nicht nur mehrere Möglichkeiten zum Erreichen einer anderen Klassifikation auf, wodurch sich die Machbarkeit erhöht, sondern lässt auch größere Rückschlüsse auf das Entscheidungsverhalten des ML-Modells schließen. 
Um Diversität zu berücksichtigen, wird in DiCE das Konzept der \textbf{D}eterminantal \textbf{P}oint \textbf{P}rocesses (DPP)  verwendet, um das \textit{Subset Selection Problem} zu lösen. Das Problem beschreibt dabei die Auswahl von wenigen CFs aus einer unendlich großen Menge an möglichen Beispielen, welche zeitgleich gültig als auch divers sind. In Gleichung \ref{dpp_diversity} beschreibt $dist(c_i,c_j)$ die Distanz zwischen zwei Counterfactuals. Somit führt eine kleine Ähnlichkeit $K_{i,j}$ der CFs zu einer großen Determinante $det(K)$ und Maximierung der Diversität.\cite{mothilal2020dice, kulesza2012determinantal}
\begin{equation}\label{dpp_diversity}
	dpp\_diversity = det(K), \text{  mit }K_{i,j} = \frac{1}{1+dist(c_i,c_j)}
\end{equation}

%- Proximity
Diversität alleine ist nicht ausreichend, um einem Anwender eine Erklärung zu geben. Die generierten CFs sollten nicht nur unterschiedlich sein, sondern müssen möglichst nah an der ursprünglichen Eingabe sein.
Diese \textbf{Proximität} ist für die Machbarkeit von zentraler Bedeutung, da Anwender den meisten Nutzen aus ähnlichen Counterfactuals erhalten. 
Die Proximität eines CFs ergibt sich aus der negative Distanz zwischen dem Counterfactual $c_i$ und dem Eingabevektor $x$. 
Eine geringe Distanz resultiert in einer hohen Proximität.
Die Berechnung der mittleren Proximität einer Menge von CFs ist in Gleichung \ref{proximity} dargestellt.\cite{mothilal2020dice}
\begin{equation}\label{proximity}
	Proximity = - \frac{1}{k} \sum_{i=1}^{k}{dist(c_i,x)}
\end{equation}

%- Sparsity
Eine weitere Eigenschaft für die Machbarkeit oder auch Umsetzbarkeit der kontrafaktischen Beispiele ist die \textbf{Sparsität}. Nach der Proximität ist auch ein Counterfactual nahe an einer Eingabe, wenn jeder Vektoreintrag minimal geändert wird. Dies ist zwar mathematisch korrekt, vernachlässigt aber den Umstand der Machbarkeit für einen Anwender. Ein Counterfactual ist einfacher umzusetzen, wenn sich möglichst wenige Eigenschaften ändern. Sparsität wird über die Anzahl an unterschiedlichen Eigenschaften zwischen Eingabe und Counterfactual gemessen.\cite{mothilal2020dice}

% User Constraints
Weiterhin ist zu beachten, dass zwar ein Counterfactual ähnlich zu dem ursprünglichen Feature-Vektor sein kann, jedoch für den Anwender nicht umsetzbar ist. Aus diesem Grund muss es die Möglichkeit geben, den Wertebereich von Eigenschaften einzuschränken und zum anderen die Änderung von Eigenschaften vollständig zu verhindern.\cite{mothilal2020dice}


\section{Bestimmung der Counterfactuals}

DiCE hat das Ziel eine Menge an $k$ Counterfactuals für einen Eingabevektor $x$ zu finden. Um die optimale Menge $C(x)$ zu ermitteln, wird die \textbf{Verlustfunktion} in Gleichung \ref{loss-function} verwendet und nach den Argumente gesucht, welche die Gesamtfunktion minimieren.\cite{mothilal2020dice}
\begin{equation}\label{loss-function}
	C(x) = \arg \min_{c_{1}, \dots, c_{k}} \underbrace{\frac{1}{k} \sum_{i=1}^{k}{yloss(f(c_i),y)}}_{\text{Term 1: Gültigkeit}} + \underbrace{\frac{\lambda_1}{k} \sum_{i=1}^{k}{dist(c_i,x)}}_{\text{Term 2: Proximität}} - \underbrace{\lambda_2 \cdot dpp\_diversity(c_1,...,c_k)}_{\text{Term 3: Diversität}}
\end{equation}
% Gültigkeit
Der erste Term beschreibt den mittleren Gültigkeitsfehler, wobei $yloss(.,.)$ die Distanz zwischen der Vorhersage $f(c_i)$ des ML-Modells für das Counterfactual $c_i$ und dem gewünschten Ergebnis $y$ misst. Die Minimierung dieses Terms führt dazu, dass die generierten Counterfactuals auch valide sind und in der gewünschten Klasse liegen.
% Proximität
Die Minimierung des zweiten Terms maximiert die Proximität der Counterfactuals $c_i$, sodass diese möglichst ähnlich zum ursprünglichen Eingabevektor $x$ sind. Über den Parameter $\lambda_1$ kann der Einfluss der Proximität auf das Gesamtergebnis variiert werden. Ein großes $\lambda_1$ die Wichtigkeit der Proximität erhöht, sodass die generierten Counterfactuals ähnlicher zu der Eingabe sind.
% Diversität
Um die Gesamtfunktion zu minimieren, muss der dritte Term maximiert werden. Die Gewichtung der Diversität wird mithilfe von $\lambda_2$ festgelegt., wobei eine Erhöhung des Parameters die Unterschiede zwischen den Counterfactuals erhöht. \cite{mothilal2020dice}
\\
\\
Die Verlustfunktion bestimmt eine optimale Menge an Counterfactuals bezüglich Gültigkeit, Proximität und Diversität, vernachlässigt jedoch die Sparsität. Aus diesem Grund erfolgt eine Nachbearbeitung der CFs, wobei alle kontinuierlichen Eigenschaften

% ERläutere Ablauf 

%\section{Alternative Ansätze} %Optional falls noch Luft ist
%- LIME vielleicht 

% was ist mit hinge loss l1 oder l2? bzw die spezifische wahl der loss funktion 					% Methodik vorstellen + Erklärung + Beispiel
\chapter{Evaluation der Methode}

In diesem Kapitel werden quantitative Evaluationsmetriken für Gültigkeit, Proximität, Diversität und Sparsität definiert, um generierte Erklärungen bewerten zu können.
% was ist mit local decisiobn bountry ???????????
Anschließend ... %todo

\section{Evaluationsmetriken}
%- Validity
Die Gültigkeitsmetrik in Gleichung \ref{validity} beschreibt den Anteil $\%ValidCFs$ an eindeutigen Counterfactuals aus der Menge $C$ aller generierten Erklärungen, wobei $k$ die Anzahl an geforderten CFs ist. Gültigkeit ist ein wichtiges Maß, da DiCE nicht auf die Einzigartigkeit der Erklärungen prüft.\cite{mothilal2020dice}
\begin{equation}\label{validity}
	\%ValidCFs = \frac{|\{\text{unique instance in } C \textit{s.t. } f(c)>0.5\}|}{k}
\end{equation}

%- Proximity
Die Proximitätsmetrik bewertet, wie nah die Counterfactuals $c_i$ an der ursprünglichen Eingabe $x$ im Feature-Raum sind. Die kontinuierlichen und kategorialen Features werden getrennt evaluiert.
In Gleichung \ref{proximity_contc} werden zunächst die Distanzen $dist\_cont$ der kontinuierlichen Features der einzelnen Counterfactuals im Bezug auf die ursprüngliche Eingabe berechnet. Anschließend wird der Mittelwert dieser Distanzen über alle $k$ generierten Counterfactuals gebildet. Abschließend wird der Wert negativ dargestellt, um die Metrik von der Distanz in einen Proximitätswert umzuwandeln, sodass eine kleine Distanz in einer großen Nähe resultiert.\cite{mothilal2020dice}
\begin{align}\label{proximity_contc}
	ContinuousProximity &:= - \frac{1}{k} \sum_{i=1}^{k}{dist\_cont(c_i,x)} %\\
					  % &= - \frac{1}{k} \sum_{i=1}^{k}{\left( \frac{1}{d_{cont}}\sum_{p=1}^{d_{cont}}\frac{|c^{p}-x^{p}|}{MAD_{p}} \right)} 
\end{align}
In Gleichung \ref{proximity_cat} wird zunächst die Distanz $dist\_cat$ der kategorialen Features der einzelnen Counterfactuals im Bezug auf die ursprüngliche Eingabe berechnet, welche zählt, wie viele kategoriale Features sich geändert haben. Anschließend wird der Mittelwert der Distanzen über alle $k$ generierten Counterfactuals gebildet. Um den Anteil der kategorialen Features zu repräsentieren, welche nicht geändert wurden, wird in einem letzten Schritt die Distanz von Eins subtrahiert. Eine hohe kategoriale Proximität resultiert somit in einem Wert nahe Eins, da nur wenige kategorialen Features geändert wurden.\cite{mothilal2020dice}
\begin{align}\label{proximity_cat}
	CategoricalProximity &:= 1 - \frac{1}{k} \sum_{i=1}^{k}{dist\_cat(c_i,x)} %\\
						% &= 1 - \frac{1}{k} \sum_{i=1}^{k}{\left( \frac{1}{d_{cat}}\sum_{p=1}^{d_{cat}}I(c^{p}\ne x^{p}) \right)}
\end{align}

%- Sparsity
Sparsität ist eine Metrik, um die Machbarkeit der generierten Erklärungen zu bewerten. Dieser Wert ist für kategoriale Features identisch mit der Proximität. Einfachheitshalber werden kontinuierliche und kategoriale Merkmale in der Sparsität zu einer einzigen Metrik in Gleichung \ref{sparsity_metric} zusammengefasst. Für jedes Counterfactual $c_i$ wird die Anzahl an veränderten Features im Vergleich zur ursprünglichen Eingabe $x$ gezählt. Jedes der $k$ Counterfactuals besitzt $d$ Merkmale. Anschließend wird der Mittelwert über alle Counterfactuals und Features berechnet und von Eins subtrahiert. Ein hoher Sparsitätswert nahe Eins bedeutet, dass nur wenige Features in den Erklärungen verändert wurden und ist für eine Umsetzbarkeit der Erklärungen anzustreben.\cite{mothilal2020dice}
\begin{equation}\label{sparsity_metric}
	Sparsity := 1- \frac{1}{kd} \sum_{i=1}^{k} \sum_{l=1}^{d}{1_{[c_i^l \ne x_i^l]}}
\end{equation}

%- Diversity
Die Diversitätsmetrik wird analog zur Proximität definiert und in kontinuierliche und kategoriale Diversität aufgeteilt, wie in den Gleichungen \ref{diversity_cont_metric} und \ref{diversity_cat_metric} dargestellt ist. Der Unterschied liegt darin, dass nun die Distanz der Features zwischen zwei Counterfactuals $c_i$ und $c_j$ gemessen wird. Die Anzahl aller möglichen CF-Paare ist $C_k^2$. Im Anschluss an die Berechnung der kategorialen und kontinuierlichen Distanz wird jeweils der Mittelwert über die Distanzen gebildet. Ein hoher Wert bedeutet, dass eine große Diversität zwischen den einzelnen Erklärungen vorliegt. Dies resultiert in größeren Unterschieden zwischen den Counterfactuals, womit eine höhere Machbarkeit einhergeht.
\cite{mothilal2020dice}
\begin{equation}\label{diversity_cont_metric}
	ContinuousDiversity := \Delta = \frac{1}{C_k^2} \sum_{i=1}^{k-1} \sum_{j=i+1}^{k}{dist\_cont(c_i,c_j)}
\end{equation}
\begin{equation}\label{diversity_cat_metric}
	CategoricalDiversity := \Delta = \frac{1}{C_k^2} \sum_{i=1}^{k-1} \sum_{j=i+1}^{k}{dist\_cat(c_i,c_j)}
\end{equation}

%- Count-Diversity
Analog zu der Sparsität kann eine weitere Diversitätsmetrik (Gleichung \ref{count-diversity}) formuliert werden, welche den Durchschnitt aller unterschiedlichen Features beim paarweisen Vergleich der Counterfactuals angibt.
\cite{mothilal2020dice}
\begin{equation}\label{count-diversity}
	CountDiversity := \Delta = \frac{1}{C_k^2d} \sum_{i=1}^{k-1} \sum_{j=i+1}^{k} \sum_{l=1}^{d}{1_{[c_i^l \ne c_j^l]}}
\end{equation}


\section{Bewertung der Erklärungsqualität}
- Wie aussagekräftig / gut sind die quantitativen Metriken?
- Wie sieht es mit qualitativen Metriken aus?
- Wie gut im Vergleich zu anderen Erkkärungsansätzen?

 	% Messbarkeit + Bewertung der Erklärungsqualität
\chapter{Praktische Anwendung}

\section{Einsatzmöglichkeiten}

\section{Demonstration eines Beispiels}

		% Anwendbarkeit in der Praxis disktuieren
%\input{chapters/WeitereKapitel}
%\input{chapters/Beispiel}
\chapter{Zusammenfassung und Kritik}

%- zusammenfassung
%- Herausforderungen 
%- wo liegen die grenzen von dice?
%- was ist problematisch und was benötigt weitere forschung
%- eigene kritische meinung, überlegungen, einschätzungen zu dice und der ergebnisse (sind ja nur begrenzt gut)

%In diesem Kapitel soll die Arbeit noch einmal kurz zusammengefasst werden. Insbesondere sollen die wesentlichen Ergebnisse Ihrer Arbeit herausgehoben werden. Erfahrungen, die z.B. Benutzer mit der Mensch-Maschine-Schnittstelle gemacht haben oder Ergebnisse von Leistungsmessungen sollen an dieser Stelle präsentiert werden. Sie können in diesem Kapitel auch die Ergebnisse oder das Arbeitsumfeld Ihrer Arbeit kritisch bewerten. Wünschenswerte Erweiterungen sollen als Hinweise auf weiterführende Arbeiten erwähnt werden.

In diesem Kapitel wird die Seminararbeit zunächst zusammengefasst und das Verfahren kritisch hinterfragt. Unter anderem werden grundlegende Probleme der Konzepte als auch fehlende Unterstützung der Anwender betrachtet. Zum Abschluss wird die eigene Meinung zum Verfahren auf Basis der zuvor genannten Kritik genannt.

\section{Zusammenfassung}
%Einleitung
In der Seminararbeit wurde DiCE \cite{mothilal2020dice} als Erklärungsansatz für Black-Box-Modelle untersucht. Zu Beginn wurde aufgezeigt, dass Entscheidungen von KI-Systemen zunehmend das Leben von Menschen beeinflussen, weshalb die Nachvollziehbarkeit und Fairness der getroffenen Entscheidungen essentiell ist. Aufgrund der Komplexität von Modellen des maschinellen Lernens sind Verfahren notwendig, die es Menschen erlauben zu verstehen wie die lokalen Entscheidungsgrenzen der Modelle verlaufen. Ein Ansatz um dieses Problem zu lösen, ist das Framework DiCE. Dazu werden zu einer unerwünschten Entscheidung Gegenbeispiele, die sogenannten kontrafaktische Erklärungen, generiert. Diese Erklärungen sollen helfen, einem Anwender aufzuzeigen, was hätte anders sein müssen, um eine gewünschte Entscheidung zu erhalten.

%Methodik
Hierzu wurden in Kapitel 2 einige grundlegende Begrifflichkeiten eingeführt, die zum Verständnis der Methodik notwendig sind.
Im Anschluss wurde das konkrete Vorgehen erläutert, wobei der Fokus auf den zentralen Konzepten: Proximität, Diversität und Sparsität lag. Proximität beschreibt die Nähe der generierten Erklärungen zu der ursprünglichen Eingabe. Nur wenn sie ähnlich sind, helfen sie dem Anwender bei der Nachvollziehbarkeit. Diversität zwischen den Counterfactuals erhöht die Machbarkeit für den Anwender, indem möglichst unterschiedliche Alternativen aufgezeigt werden. Diese beiden Konzepte werden in der Verlustfunktion kombiniert, auf deren Basis die Erklärungen generiert werden. Sparsität erhöht die Machbarkeit in einem Nachbearbeitungsschritt noch weiter, indem die Anzahl der veränderten Features zwischen ursprünglicher Eingabe und Counterfactual minimiert wird.

%Evaluation
Um die Ergebnisse bewerten zu können, wurden Metriken zu den Konzepten eingeführt. Als Vergleichsmethoden wurden vier Baselines und LIME \cite{Ribeiro.2016} verwendet und DiCE gegenüber gestellt. Es zeigte sich, dass DiCE in den verwendeten Datensätzen bessere Resultate als LIME und ähnliche Werte wie die Baselines erzielte. Bei der Proximität hingegen schneiden die Baselines teilweise besser ab als DiCE, da dieses durch die Diversität eine geringere Proximität erzielt wird.

%Demo
Zum Abschluss wurde DiCE an einem konkreten Beispiel demonstriert, welches von den Autoren zur Verfügung gestellt wird. Es wurde detailliert erläutert, wie die theoretischen Konzepte programmatisch umgesetzt wurden und wie DiCE in der Praxis angewendet werden kann. Weiterhin wurde auf die Anwendungsmöglichkeiten in der Praxis eingegangen und wie das Verfahren Anwendern im Alltag unterstützen kann.


\section{Kritische Betrachtung}
%Kritik und Herausforderungen von DiCE
Der Erklärungsansatz von DiCE, durch Gegenbeispiele einem Anwender die Entscheidung nachvollziehen zu lassen, ist sehr einfach zu verstehen und daher praktisch breit anwendbar. Die Methodik weist jedoch einige Probleme auf. 

%Machbarkeitsproblem durch manuelle Limitierung der änderbaren Features
Die Machbarkeit wird durch die Diversität und Sparsität verbessert, jedoch wird ein wichtiger Schritt an den Anwender abgegeben. DiCE macht keine direkten Einschränkungen, welche Features der Eingabe änderbar sein sollen \cite{mothilal2020dice}. Das Verfahren behandelt zum Beispiel die Änderung der ethnischen Zugehörigkeit, die Verringerung des Alters als auch die Änderung des Wohnorts als gleichwertig. Das bedeutet, dass die Einschränkung der änderbaren Features und die Festlegung von Intervallen in denen die Werte verändert werden dürfen, durch den Anwender jedes mal selbst festgelegt werden muss. Dies ist zeitintensiv und erfordert unter Umständen tiefes Verständnis der betrachteten Domäne und des Individuums. Beispielsweise wird ein Kredit aufgrund des Wohnorts abgelehnt. Ein Umzug ist aus familiären oder finanziellen Gründen für die eine Person nicht möglich, aber für die nächste Person unbedenklich und der einfachste Weg, damit das Modell den Kredit gewährt.

%Zielkonflikt zwischen Proximität und Diversität
Ein zweites Problem ist die Verlustfunktion selbst. Aufgrund der konkurrierenden Zielen von Proximität und Diversität, werden diese beiden Konzepte in der Verlustfunktion durch die beiden $\lambda$-Parameter gewichtet \cite{mothilal2020dice}. Sollen die Erklärungen näher an der ursprünglichen Eingabe liegen, um die Relevanz für den Anwender zu erhöhen, wird automatisch die Distanz zwischen den Erklärungen verringert, sodass die Counterfactuals sich weniger unterscheiden. Dies reduziert aber die Wahrscheinlichkeit, dass der Anwender eine machbare Alternative findet.

%Kognitive Überlastung durch zu viele Erklärungen
Als letztes Problem ist die Anzahl der Counterfactuals zu betrachten. DiCE bietet die Möglichkeit viele Erklärungen zu generieren. Hierbei stellt sich die Frage, wie viele Beispiele ein Anwender überhaupt überblicken und verarbeiten kann. Unter Betrachtung des vorherigen Kreditbeispiels sind vermutlich 1000 Änderungsmöglichkeiten zum Erhalt des Kredits für einen Anwender überfordernd. Dieser Punkt der kognitiven Überlastung durch eine zu große Menge an Informationen wurde durch die Autoren nicht näher betrachtet \cite{mothilal2020dice}. Die Aufgabe der optimalen Anzahl wird erneut dem Anwender selbst überlassen.
\\
\\
%Eigene Meinung
Meiner Meinung nach stellt DiCE einen guten Ansatz dar, um komplexe Modelle für einen Anwender verständlich und nachvollziehbar zu machen. Die Verwendung von Beispielen als Erklärung ist intuitiv und leicht verständlich, auch ohne technisches Verständnis und lässt einen Einblick in die Entscheidungsgrenzen des Modells zu. Weiterhin ist der Anwender nicht auf eine einzelne Erklärung beschränkt, sondern kann sich die beste Variante heraussuchen. Verbesserungswürdig ist die umfangreiche Vorarbeit durch den Anwender die notwendig ist, um die sinnvolle Erklärungen zu erhalten. Dies reduziert die Anwendung in der Praxis, insbesondere für den unkomplizierten Einsatz durch einen Laien. Eine Unterstützung durch Sortierung der Erklärungen nach Machbarkeit wäre wünschenswert, um die Verwendbarkeit von DiCE zu erhöhen. Wie die Autoren selbst angemerkt haben ist auch die Aussagekraft der Paper-Ergebnisse limitiert, da keine Verhaltensstudie zur Bewertung der Entscheidungsgrenze mit Menschen durchgeführt wurde.





 


%------------------ Literaturverzeichnis & Index -------------------------------
\backmatter
\bibliography{literatur}								% Literaturverzeichnis (literatur.bib)
\printindex												% Index (optional)


%------------------ Anhänge ----------------------------------------------------
\begin{appendix}
%	\include{chapters/Glossar}							% Glossar (optional)
	\chapter{Eigenständigkeitserklärung}

% Falls die vorliegende Arbeit als Gruppenarbeit angefertigt wurde, muss diese Eigenständigkeitserklärung dupliziert werden und von jedem auf dem Deckblatt angegebenen Bearbeiter separat ausgefüllt werden.

\begin{small}

\begin{description}
\item[$\Box$] Die vorliegende Arbeit wurde als Einzelarbeit angefertigt.\\

\item[$\Box$] Die vorliegende Arbeit wurde als Gruppenarbeit angefertigt. Mein Anteil an der Gruppenarbeit ist im untenstehenden Abschnitt \emph{Verantwortliche} dokumentiert:\\

\vspace{1cm}

\item[$\Box$] Hiermit erkläre ich, dass ich die vorliegende Arbeit selbstständig und ohne unzulässige Hilfe Dritter angefertigt habe. Ich habe keine anderen als die angegebenen Quellen und Hilfsmittel benutzt sowie wörtliche und sinngemäße Zitate als solche kenntlich gemacht. Darüber hinaus erkläre ich, dass ich die vorliegende Arbeit in dieser oder ähnlicher Form noch nicht als Prüfungsleistung eingereicht habe.\\

\vspace{1cm}

\item[$\Box$] Es ist keine Nutzung von KI-basierten text- oder inhaltgenerierenden Hilfsmitteln erfolgt.\\

\item[$\Box$] Die Nutzung von KI-basierten text- oder inhaltgenerierenden Hilfsmitteln wurde von der/dem Prüfenden ausdrücklich gestattet. Die von der/dem Prüfenden mit Ausgabe der Arbeit vorgegebenen Anforderungen zur Dokumentation und Kennzeichnung habe ich erhalten und eingehalten. Sofern gefordert, habe ich in der untenstehenden Tabelle \emph{Nutzung von KI-Tools} die verwendeten KI-basierten text- oder inhaltgenerierenden Hilfsmittel aufgeführt und die Stellen in der Arbeit genannt. Die Richtigkeit übernommener KI-Aussagen und Inhalte habe ich nach bestem Wissen und Gewissen überprüft.\\
\end{description}

\vspace{4cm}
\begin{minipage}[t]{3cm}
	\rule{3cm}{0.5pt}
	Datum
\end{minipage}
\hfill
\begin{minipage}[t]{9cm}
	\rule{9cm}{0.5pt}
	Unterschrift der Kandidatin/des Kandidaten
\end{minipage}

\end{small}

\newpage

%\section*{Verantwortliche}
%
%Die Tabellen unten führen auf, wer als Autor für die einzelnen Kapitel der vorliegenden Dokumentation beziehungsweise für einzelne Teile des Quellcodes hauptverantwortlich ist.
%
%Insgesamt beteiligt sind die folgenden Personen:
%
%\begin{itemize}
%	\item Fabian Wagner
%\end{itemize}
%
%\subsection*{Dokumentation}
%
%\begin{table}
%	\begin{small}
%	\begin{tabularx}{\textwidth}{|l|X|X|}
%		\hline		
%		\textbf{Kapitel} & \textbf{Überschrift} & \textbf{Autor}\\
%		\hline
%		Alles & Alles & Fabian Wagner\\
%%		\hline
%%		2 & Problemstellung & Autor 1, Autor 2, Autor 3\\
%%		\hline
%%		3 & Aufgabenstellung und Zielsetzung & Autor 1, Autor 2, Autor 3\\
%%		\hline
%%		4 & Übrige Abschnitte (Kapitel und Absätze) & Autor 1\\
%%		\hline
%%		4.1 & Abschnitt & Autor 3\\
%%		\hline
%%		4.1.1 & Unterabschnitt & Autor 2, Autor 3\\
%%		\hline
%%		4.2 & Abbildungen und Tabellen & usw.\\
%%		\hline
%%		4.3 & Mathematische Formel	&  \\
%%		\hline
%%		4.4 & Sätze, Lemmas und Definitionen &  \\
%%		\hline
%%		4.5 & Fußnoten & \\
%%		\hline
%%		4.6 & Literaturverweise & \\
%%		\hline
%%		5 & Beispiel-Kapitel & \\
%%		\hline
%%		5.1 & Warum existieren unterschiedliche Konsistenzmodelle? & \\
%%		\hline
%%		5.2 & Klassifizierung eines Konsistenzmodells	& \\
%%		\hline
%%		5.3 & Linearisierbarkeit (atomic consistency)	& \\
%		\hline
%	\end{tabularx}
%	\end{small}
%\end{table}
%
%%\subsection*{Quellcode}
%
%\begin{table}
%	\begin{small}
%	\begin{tabularx}{\textwidth}{|X|X|}
%		\hline		
%		\textbf{Paket} & \textbf{Autor}\\
%		\hline
%		algorithms.search & Autor 1\\
%		\hline
%		algorithms.sort & Autor 3\\
%		\hline
%	\end{tabularx}
%	\end{small}
%\end{table}

%\newpage

\subsection*{Nutzung von KI-Tools}

\begin{table}
	\begin{small}	
	\begin{tabularx}{\textwidth}{|X|X|X|X|X|X|}
		\hline		
		\textbf{KI-Tool} & \textbf{Genutzt für} & \textbf{Warum?} & \textbf{Wann?} & \textbf{Mit welcher Eingabefrage bzw. -aufforderung?} & \textbf{An welcher Stelle der Arbeit übernommen?}\\
		\hline
		ChatGPT, DeepSeek, Gemini & Korrektur bzgl. Rechtschreibung, Grammatik und Formulierungen & Verbesserung der Textqualität & Während der gesamten Arbeit & Textabschnitte mit der Aufforderung zur Kontrolle & Über die gesamte Arbeit hinweg \\
		\hline
		Gemini & Erstellung der Abbildungen \ref{fig:ml-prozess} und \ref{fig:ml-prozess-cf} & Schnelle und einfache Modifikation durch Latex-Code & Für die Abbildungen  \ref{fig:ml-prozess} und \ref{fig:ml-prozess-cf} & Erstelle ein Bild in Latex mit dem Inhalt: input [200€, Männlich, Arbeitslos] -> Black-Box -> "Kein Kredit" als Beispiel zur Erläuterung der Datenverarbeitung eines Black-Box-Modells.  &  Abbildungen \ref{fig:ml-prozess} und \ref{fig:ml-prozess-cf} \\
		\hline
		Gemini & Anfertigung des Evaluationsskripts & Autoren haben keine eigene Evaluation bereitgestellt, daher wurde diese mit Hilfe von Gemini angefertigt, für eine bessere Diskussion der Ergebnisse. & Für die Werte in Tabelle \ref{tab:evaluation_comparison} & Als Eingabe wurden die Metriken und die CF-Mengen der Demonstration verwendet. & Tabelle \ref{tab:evaluation_comparison} \\
		\hline
%		DeepL Write & Neuformulierung meiner Textentwürfe & Bessere Lesbarkeit & Über die gesamte Arbeit hinweg & Formuliere die Kapitel 2 und 3 neu in einfachen und leicht verständlichen Sätzen! & S. 45 ff. , S. 67\\
%		\hline
	\end{tabularx}
	\end{small}
\end{table}

		% Eigenständigkeitserklärung
\end{appendix}


\end{document}
