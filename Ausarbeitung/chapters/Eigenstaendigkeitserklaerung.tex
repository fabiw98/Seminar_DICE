\chapter{Eigenständigkeitserklärung}

% Falls die vorliegende Arbeit als Gruppenarbeit angefertigt wurde, muss diese Eigenständigkeitserklärung dupliziert werden und von jedem auf dem Deckblatt angegebenen Bearbeiter separat ausgefüllt werden.

\begin{small}

\begin{description}
\item[$\Box$] Die vorliegende Arbeit wurde als Einzelarbeit angefertigt.\\

\item[$\Box$] Die vorliegende Arbeit wurde als Gruppenarbeit angefertigt. Mein Anteil an der Gruppenarbeit ist im untenstehenden Abschnitt \emph{Verantwortliche} dokumentiert:\\

\vspace{1cm}

\item[$\Box$] Hiermit erkläre ich, dass ich die vorliegende Arbeit selbstständig und ohne unzulässige Hilfe Dritter angefertigt habe. Ich habe keine anderen als die angegebenen Quellen und Hilfsmittel benutzt sowie wörtliche und sinngemäße Zitate als solche kenntlich gemacht. Darüber hinaus erkläre ich, dass ich die vorliegende Arbeit in dieser oder ähnlicher Form noch nicht als Prüfungsleistung eingereicht habe.\\

\vspace{1cm}

\item[$\Box$] Es ist keine Nutzung von KI-basierten text- oder inhaltgenerierenden Hilfsmitteln erfolgt.\\

\item[$\Box$] Die Nutzung von KI-basierten text- oder inhaltgenerierenden Hilfsmitteln wurde von der/dem Prüfenden ausdrücklich gestattet. Die von der/dem Prüfenden mit Ausgabe der Arbeit vorgegebenen Anforderungen zur Dokumentation und Kennzeichnung habe ich erhalten und eingehalten. Sofern gefordert, habe ich in der untenstehenden Tabelle \emph{Nutzung von KI-Tools} die verwendeten KI-basierten text- oder inhaltgenerierenden Hilfsmittel aufgeführt und die Stellen in der Arbeit genannt. Die Richtigkeit übernommener KI-Aussagen und Inhalte habe ich nach bestem Wissen und Gewissen überprüft.\\
\end{description}

\vspace{4cm}
\begin{minipage}[t]{3cm}
	\rule{3cm}{0.5pt}
	Datum
\end{minipage}
\hfill
\begin{minipage}[t]{9cm}
	\rule{9cm}{0.5pt}
	Unterschrift der Kandidatin/des Kandidaten
\end{minipage}

\end{small}

\newpage

%\section*{Verantwortliche}
%
%Die Tabellen unten führen auf, wer als Autor für die einzelnen Kapitel der vorliegenden Dokumentation beziehungsweise für einzelne Teile des Quellcodes hauptverantwortlich ist.
%
%Insgesamt beteiligt sind die folgenden Personen:
%
%\begin{itemize}
%	\item Fabian Wagner
%\end{itemize}
%
%\subsection*{Dokumentation}
%
%\begin{table}
%	\begin{small}
%	\begin{tabularx}{\textwidth}{|l|X|X|}
%		\hline		
%		\textbf{Kapitel} & \textbf{Überschrift} & \textbf{Autor}\\
%		\hline
%		Alles & Alles & Fabian Wagner\\
%%		\hline
%%		2 & Problemstellung & Autor 1, Autor 2, Autor 3\\
%%		\hline
%%		3 & Aufgabenstellung und Zielsetzung & Autor 1, Autor 2, Autor 3\\
%%		\hline
%%		4 & Übrige Abschnitte (Kapitel und Absätze) & Autor 1\\
%%		\hline
%%		4.1 & Abschnitt & Autor 3\\
%%		\hline
%%		4.1.1 & Unterabschnitt & Autor 2, Autor 3\\
%%		\hline
%%		4.2 & Abbildungen und Tabellen & usw.\\
%%		\hline
%%		4.3 & Mathematische Formel	&  \\
%%		\hline
%%		4.4 & Sätze, Lemmas und Definitionen &  \\
%%		\hline
%%		4.5 & Fußnoten & \\
%%		\hline
%%		4.6 & Literaturverweise & \\
%%		\hline
%%		5 & Beispiel-Kapitel & \\
%%		\hline
%%		5.1 & Warum existieren unterschiedliche Konsistenzmodelle? & \\
%%		\hline
%%		5.2 & Klassifizierung eines Konsistenzmodells	& \\
%%		\hline
%%		5.3 & Linearisierbarkeit (atomic consistency)	& \\
%		\hline
%	\end{tabularx}
%	\end{small}
%\end{table}
%
%%\subsection*{Quellcode}
%
%\begin{table}
%	\begin{small}
%	\begin{tabularx}{\textwidth}{|X|X|}
%		\hline		
%		\textbf{Paket} & \textbf{Autor}\\
%		\hline
%		algorithms.search & Autor 1\\
%		\hline
%		algorithms.sort & Autor 3\\
%		\hline
%	\end{tabularx}
%	\end{small}
%\end{table}

%\newpage

\subsection*{Nutzung von KI-Tools}

\begin{table}
	\begin{small}	
	\begin{tabularx}{\textwidth}{|X|X|X|X|X|X|}
		\hline		
		\textbf{KI-Tool} & \textbf{Genutzt für} & \textbf{Warum?} & \textbf{Wann?} & \textbf{Mit welcher Eingabefrage bzw. -aufforderung?} & \textbf{An welcher Stelle der Arbeit übernommen?}\\
		\hline
		ChatGPT, DeepSeek, Gemini & Korrektur bzgl. Rechtschreibung, Grammatik und Formulierungen & Verbesserung der Textqualität & Während der gesamten Arbeit & Textabschnitte mit der Aufforderung zur Kontrolle & Über die gesamte Arbeit hinweg \\
		\hline
		Gemini & Erstellung der Abbildungen \ref{fig:ml-prozess} und \ref{fig:ml-prozess-cf} & Schnelle und einfache Modifikation durch Latex-Code & Für die Abbildungen  \ref{fig:ml-prozess} und \ref{fig:ml-prozess-cf} & Erstelle ein Bild in Latex mit dem Inhalt: input [200€, Männlich, Arbeitslos] -> Black-Box -> "Kein Kredit" als Beispiel zur Erläuterung der Datenverarbeitung eines Black-Box-Modells.  &  Abbildungen \ref{fig:ml-prozess} und \ref{fig:ml-prozess-cf} \\
		\hline
		Gemini & Anfertigung des Evaluationsskripts & Autoren haben keine eigene Evaluation bereitgestellt, daher wurde diese mit Hilfe von Gemini angefertigt, für eine bessere Diskussion der Ergebnisse. & Für die Werte in Tabelle \ref{tab:evaluation_comparison} & Als Eingabe wurden die Metriken und die CF-Mengen der Demonstration verwendet. & Tabelle \ref{tab:evaluation_comparison} \\
		\hline
%		DeepL Write & Neuformulierung meiner Textentwürfe & Bessere Lesbarkeit & Über die gesamte Arbeit hinweg & Formuliere die Kapitel 2 und 3 neu in einfachen und leicht verständlichen Sätzen! & S. 45 ff. , S. 67\\
%		\hline
	\end{tabularx}
	\end{small}
\end{table}

