\chapter{Evaluation der Methode}

In diesem Kapitel werden zunächst qualitative Evaluationsmetriken für Gültigkeit, Proximität, Diversität und Sparsität definiert.
% was ist mit local decisiobn bountry ???????????
Anschließend ... %todo

\section{Evaluationsmetriken}
%- macht dice auch das was es soll (löst es das problem)
%- wie gut ist es?
%- wie lauten die Metriken welche zur auswertung verwendet wurden?
%- sind die metriken sinnvoll und gut?

%- Validity
Die Gültigkeitsmetrik ist der Anteil der eindeutigen Counterfactuals aus der generierten Menge $C$.\cite{mothilal2020dice}
\begin{equation}\label{validity}
	\%ValidCFs = \frac{|\{unique instances in C s.t. f(c)>0.5\}|}{k}
\end{equation}

%- Proximity

\begin{equation}\label{proximity_contc}
	ContinousProximity: - \frac{1}{k} \sum_{i=1}^{k}{dist\_cont(c_i,x)}
\end{equation}
\begin{equation}\label{proximity_cat}
	CategoricalProximity: 1 - \frac{1}{k} \sum_{i=1}^{k}{dist\_cat(c_i,x)}
\end{equation}

- Sparsity
\begin{equation}\label{sparsity_metric}
	Sparsity: 1- \frac{1}{kd} \sum_{i=1}^{k} \sum_{l=1}^{d}{1_{[c_i^l \ne x_i^l]}}
\end{equation}

- Diversity
\begin{equation}\label{diversity_metric}
	Diversity: \Delta = \frac{1}{C_k^2} \sum_{i=1}^{k-1} \sum_{j=i+1}^{k}{dist(c_i,c_j)}
\end{equation}

- Count-Diversity
\begin{equation}\label{count-diversity}
	Count-Diversity: \Delta = \frac{1}{C_k^2d} \sum_{i=1}^{k-1} \sum_{j=i+1}^{k} \sum_{l=1}^{d}{1_{[c_i^l \ne c_j^l]}}
\end{equation}


\section{Bewertung}
- ist DICE besser als zb LIME?
- Gründe für die Unterschiede

