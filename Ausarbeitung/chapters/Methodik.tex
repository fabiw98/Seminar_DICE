\chapter{Methodisches Vorgehen}
In diesem Kapitel werden die zentralen Konzepte von DiCE betrachtet, welche zur Generierung der Counterfactuals benötigt werden. Darunter fallen Proximität, Diversität und Sparsität. Abschließend wird die Verlustfunktion als Möglichkeit der Optimierung von Counterfactuals untersucht.


\section{Zentrale Konzepte}
%- Diversity via dpp
\textbf{Diversität} beschreibt, wie sich generierten Counterfactuals voneinander unterscheiden.
Eine hohe Vielfältigkeit zeigt dem Anwender nicht nur mehrere Möglichkeiten zum Erreichen einer anderen Klassifikation auf, wodurch sich die Machbarkeit erhöht, sondern lässt auch größere Rückschlüsse auf das Entscheidungsverhalten des ML-Modells schließen. 
Um Diversität zu berücksichtigen, wird in DiCE das Konzept der \textbf{D}eterminantal \textbf{P}oint \textbf{P}rocesses (DPP)  verwendet, um das \textit{Subset Selection Problem} zu lösen. Das Problem beschreibt dabei die Auswahl von wenigen CFs aus einer unendlich großen Menge an möglichen Beispielen, welche zeitgleich gültig als auch divers sind. In Gleichung \ref{dpp_diversity} beschreibt $dist(c_i,c_j)$ die Distanz zwischen zwei Counterfactuals. Somit führt eine kleine Ähnlichkeit $K_{i,j}$ der CFs zu einer großen Determinante $det(K)$ und Maximierung der Diversität. \cite{mothilal2020dice, kulesza2012determinantal}
\begin{equation}\label{dpp_diversity}
	dpp\_diversity = det(K), \text{  mit }K_{i,j} = \frac{1}{1+dist(c_i,c_j)}
\end{equation}

%- Proximity
Diversität alleine ist nicht ausreichend, um einem Anwender eine Erklärung zu geben. Die generierten CFs sollten nicht nur unterschiedlich sein, sondern müssen möglichst nah an der ursprünglichen Eingabe sein.
Diese \textbf{Proximität} ist für die Machbarkeit von zentraler Bedeutung, da Anwender den meisten Nutzen aus ähnlichen Counterfactuals erhalten. 
Die Proximität eines CFs ergibt sich aus der negative Distanz zwischen dem Counterfactual $c_i$ und dem Feature-Vektor $x$. 
Eine geringe Distanz resultiert in einer hohen Proximität.
Die Berechnung der mittleren Proximität einer Menge von CFs ist in Gleichung \ref{proximity} dargestellt.
\begin{equation}\label{proximity}
	Proximity = - \frac{1}{k} \sum_{i=1}^{k}{dist(c_i,x)}
\end{equation}

%- Sparsity
Eine weitere Eigenschaft für die Machbarkeit oder auch Umsetzbarkeit der kontrafaktischen Beispiele ist die \textbf{Sparsität}. Nach der Proximität ist auch ein Counterfactual nahe an einer Eingabe, wenn jeder Vektoreintrag minimal geändert wird. Dies ist zwar mathematisch korrekt, vernachlässigt aber den Umstand der Machbarkeit für einen Anwender. Ein Counterfactual ist einfacher umzusetzen, wenn sich möglichst wenige Eigenschaften ändern. Sparsität wird über die Anzahl an unterschiedlichen Eigenschaften zwischen Eingabe und Counterfactual gemessen.


\section{Optimierung}
% Verlustfunktion wird hier erläutert
s
\begin{equation}\label{loss-function}
	C(x) = arg min \frac{1}{k} \sum_{i=1}^{k}{yloss(f(c_i),y)} + \frac{\lambda_1}{k} \sum_{i=1}^{k}{dist(c_i,x)} - \lambda_2  dpp_diversity(c_1,...,c_k)
\end{equation}

% ERläutere Ablauf 

%\section{Alternative Ansätze} %Optional falls noch Luft ist
%- LIME vielleicht 