\chapter{Einleitung und Problemstellung}

%todo prüfe welche englischen begriffe übersetzt werden sollen!!!!
%todo ausfürhlicher mit beispielen
%todo einarbeiten des Papers 0 in die Einleitung und zitieren aus diesem!
Künstliche Intelligenz etabliert sich zu einem festen Bestandteil der Arbeitswelt und dem Alltag. Neben dem Einsatz als Chatbot oder Assisstenten treffen KI-Systeme vermehrt Entscheidungen, welche folgenschwere Konsequenzen auf das Leben haben. Die Nachvollziehbarkeit einer solchen Entscheidung ist für den sicheren und fairen Einsatz von KIs von zentraler Bedeutung. Erklärbare KI (xAI) befasst sich mit der Fragestellung die getroffene Klassifikation einer KI für den Menschen verständlich und nachvollziehbar zu machen.

Ziel der Seminararbeit ist die Analyse und Evaluation der Methodik von DiCE \cite{mothilal2020dice} sowie die Bewertung für die Anwendbarkeit der Methode in der Praxis. Der Ansatz von DiCE beruht auf der Erklärung durch Beispiele, die ähnlich zur ursprünglichen Eingabe sind, jedoch nicht zur gleichen Klassifikation durch die KI führen. Als erstes werden grundlegende Begriffe erklärt, welche für das Verständnis der Arbeit notwendig sind. Anschließend wird das methodische Vorgehen von DiCE untersucht. Hierzu werden die Konzepte Diversity, Proximity, Sparsity, User Constraints sowie Feasibility erläutert und deren Abhängigkeiten betrachtet. Die Bewertung von DiCE erfolgt mithilfe von Evaluationsmetriken für die einzelnen Konzepte, welche zuvor definiert werden. Weiterhin wird ein Beispiel demonstriert, an dem die Verständlichkeit einer solchen Erklärung durch Counterfactuals verdeutlicht werden soll. Abschließend folgt eine Zusammenfassung und kritische Diskussion der vorgestellten Methodik.


%
%Begonnen werden soll mit einer Einleitung zum Thema, also Hintergrund und Ziel erläutert werden.
%
%Weiterhin wird das vorliegende Problem diskutiert: Was ist zu lösen, warum ist es wichtig, dass man dieses Problem löst und welche Lösungsansätze gibt es bereits. Der Bezug auf vorhandene oder eben bisher fehlende Lösungen begründet auch die Intention und Bedeutung dieser Arbeit. Dies können allgemeine Gesichtspunkte sein: Man liefert einen Beitrag für ein generelles Problem oder man hat eine spezielle Systemumgebung oder ein spezielles Produkt (z.B. in einem Unternehmen), woraus sich dieses noch zu lösende Problem ergibt.
%
%Im weiteren Verlauf wird die Problemstellung konkret dargestellt: Was ist spezifisch zu lösen? Welche Randbedingungen sind gegeben und was ist die Zielsetzung? Letztere soll das beschreiben, was man mit dieser Arbeit (mindestens) erreichen möchte.
